%!TEX program = xelatex
\documentclass[11pt, a4paper]{article}
\usepackage{cite}
\usepackage{ctex}
\usepackage{natbib}
\setcitestyle{authoryear, open={(}, close = {)}}
\usepackage{amsmath, amssymb}
\usepackage{graphicx}
\usepackage{array}
\usepackage{booktabs}
\usepackage{enumerate}
\usepackage{bm}
\usepackage{color}
\usepackage{pifont}
\usepackage[final]{pdfpages}

\usepackage{geometry}
\geometry{left = 2cm, right = 2.4cm, top = 2.4cm, bottom = 3cm}
\renewcommand{\baselinestretch}{1.5}
\usepackage{fancyhdr, fancybox}
\usepackage{lastpage}
\pagestyle{fancy}
\lhead{}
\chead{}
\rhead{}
\cfoot{}
\rfoot{}
\lfoot{}
\cfoot{\fontsize{10.5pt}{\baselineskip}\selectfont 第\ \thepage\ 页 \quad 共\ \pageref{LastPage}\ 页}
\renewcommand{\headrulewidth}{0cm}
\renewcommand{\footrulewidth}{0cm}

\begin{document}

\thispagestyle{empty}
\newgeometry{left = 1cm, right = 1cm, top = 3cm, bottom = 3cm}
\begin{center}
  \includegraphics[width=0.6\textwidth]{badge.png}
\end{center}

\vskip 0.5cm
\begin{table}[!h]
  \setlength\tabcolsep{0cm}
  \renewcommand{\arraystretch}{1}
  \renewcommand{\CJKglue}{\hskip 6pt plus 0.08\baselineskip}
  \begin{center}
    \fontsize{36pt}{\baselineskip}\selectfont
    \begin{tabular}{p{18cm}<{\centering}}
    硕士学位论文开题报告 \\
    \midrule
    \bottomrule[3pt]
    \end{tabular}
  \end{center}
\end{table}

\vskip 2cm
\begin{center}
  \fontsize{22pt}{\baselineskip}\selectfont
  建议用xelatex编译
\end{center}

\vskip 3cm
\begin{table}[!h]
  \setlength\tabcolsep{4pt}
  \renewcommand{\arraystretch}{0.5}
  \begin{center}
    \fontsize{18pt}{\baselineskip}\selectfont
    \begin{tabular}{p{3.2cm} p{4.8cm}<{\centering}}
      \textbf{学生姓名:} & \textbf{姓名} \\
      \cline{2-2}
      \vskip 0.1cm \textbf{学号:} & \vskip 0.2cm \textbf{学号} \\
      \cline{2-2}
      \vskip 0.1cm \textbf{指导教师:} & \vskip 0.1cm \textbf{姓名} \\
      \cline{2-2}
      \vskip 0.1cm \textbf{专业:} & \vskip 0.1cm \textbf{专业} \\
      \cline{2-2}
      \vskip 0.1cm \textbf{院系:} & \vskip 0.1cm \textbf{学院} \\
      \cline{2-2}
    \end{tabular}
  \end{center}
\end{table}

\vskip 2cm
\begin{center}
  \fontsize{18pt}{\baselineskip}\selectfont
  \renewcommand{\CJKglue}{\hskip 6pt plus 0.08\baselineskip}
  \textbf{
    北京航空航天大学XX学院 \\
    \today
  }
\end{center}

\restoregeometry
\newpage
\setcounter{page}{1}

\fancypage{%
  \setlength{\fboxsep}{0.2cm}%
  \setlength{\fboxrule}{0.8pt}%
  \setlength{\shadowsize}{0cm}%
  \shadowbox}
{}

\vspace*{-0.6cm}
\noindent
{\fontsize{12pt}{\baselineskip}\selectfont \bfseries
  %一、论文选题依据(包括论文选题的意义、国内外研究现状分析等)
  一、论文选题的背景意义和根据 
}
\newline
(对基础研究,着重结合国际科学发展趋势,论述课题的科学意义;对应用基础或应用研究,着重结合学科前沿、围绕国民经济、社会发展和管理实践的重要问题,论述其应用前景)

\vskip 0.2cm

这个模板只是一个参考,能不能使用大家还是要看自己学院教务的要求(如果需要提交word版本的话,那还是别用 \LaTeX ),当然大家可以在这个上面进行修改。

前人栽树后人乘凉,感谢创作这个模板的前辈,我替换了一个校徽校名的图片,原来的实在是太糊了,还有白色阴影,估计是从别的地方扣下来的,我更换了一个AI导出的版本,当然这个校徽加校名的图片尺寸也咩有很大,但已经够用了,另外附上一个更清楚的(big\_badge.png)。

文献综述的话,大家可以自行用这个模板改。

\newpage

\noindent
{\fontsize{12pt}{\baselineskip}\selectfont \bfseries
	二、国内外研究现状及发展动态分析 
}
\newline
(包括国内外研究现状和实践发展状况分析)

\noindent
{\bfseries
(一) XXX的相关研究
}

小标题是我随便编的,不是模版上带的,下一个章节同理。

\vskip 0.1cm
\noindent
{\bfseries
(二) XXX的相关概述
}

\vskip 0.1cm
\noindent
{\bfseries
(三) XXX的估计方法
}

\vskip 0.1cm
\noindent
{\bfseries
(四) XXX的相关研究
}

\vskip 0.1cm
\noindent
{\bfseries
(五) 结论
}


\newpage
\noindent
{\fontsize{12pt}{\baselineskip}\selectfont \bfseries
  三、课题的研究内容及拟采用的研究方法、技术路线及研究难点,预期达到的目标
}
\newline
(1. 课题的研究内容;2. 拟采用的研究方法、技术路线及研究难点;3. 预期达到的目标。)

\vskip 0.2cm

\vskip 0.1cm
\noindent
{\bfseries
  1.课题的研究内容
}

\vskip 0.1cm
\noindent
{\bfseries
  2.技术路线
}

\vskip 0.1cm
\noindent
{\bfseries
  3.创新点
}

\vskip 0.1cm
\noindent
{\bfseries
  4.研究难点
}

\vskip 0.1cm
\noindent
{\bfseries
  5.预期达到的目标
}


\newpage

\noindent
{\fontsize{12pt}{\baselineskip}\selectfont \bfseries
  四、论文详细工作进度安排
}
\newline


\begin{table}[!h]
  \setlength\tabcolsep{5pt}
  \begin{center}
    \begin{tabular}{p{5.5cm} p{9.5cm}}
    2022 年\ 9 月\ -- 2022 年\ 10 月\ & 撰写开题报告和文献综述,完成开题\\
    2022 年\ 11 月\ & 研究XXX的构建理论方法 \\
    2022 年\ 12 月\ -- 2023 年\ 1月\ & 构建XXX模型 \\
    2023 年\ 2 月\ & 算法设计及有效性证明 \\
    2023 年\ 3 月\ & 算法编程实现 \\
    2023 年\ 4 月\ -- 2023 年\ 5 月\ & 研究XXX的XXX \\
    2023 年\ 6 月\ -- 2021 年\ 8 月\ & 实例应用及模型解释,撰写论文 \\
    2023 年\ 9 月\ & 完成中期检查,完成论文初稿\\
    2023 年\ 10 月\ -- 2023 年\ 11 月\ & 完成论文,准备答辩 \\
    \end{tabular}
  \end{center}
\end{table}
\vspace*{-0.6cm}

\newpage
\bibliographystyle{plainnat}
\bibliography{bibs}
\end{document}
